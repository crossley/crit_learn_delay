\documentclass[10pt]{article}
\usepackage{graphicx}
\usepackage{parskip}
\usepackage{geometry}
\geometry{
  top=20mm,
}
\usepackage{fancyhdr}
\lhead{\includegraphics[height=2.0cm]{../figures/MQ_MAS_HOR_RGB_POS.png}\vspace{0.0cm}}
\rhead{\includegraphics[width=6.0cm]{../figures/UC_Santa_Barbara_Wordmark_Navy_RGB.png}\vspace{0.5cm}}
\setlength\headheight{2cm}

\usepackage[hang, symbol]{footmisc}
\setlength\footnotemargin{10pt}
\renewcommand{\thefootnote}{\fnsymbol{footnote}}

\usepackage{hyperref}

\pagenumbering{gobble}

\renewcommand{\thefootnote}{\dag}

\begin{document}
\thispagestyle{fancy}
Dear Editors,

I am pleased to submit our manuscript titled ``Criterial
Learning and Feedback Delay: Insights from Computational
Models and Behavioral Experiments'' for consideration in the
\textit{Journal of Experimental Psychology: Learning,
Memory, and Cognition}.

In this work, we develop and evaluate three novel
computational models designed to explore the mechanisms
underlying criterial learning. We test the predictions
of these models in two behavioral experiments, focusing on
the effects of feedback delay and intertrial intervals (ITI)
on criterial learning in both models and humans.

Our results indicate that human criterial learning is
significantly impaired by delayed feedback, but not by
extended ITIs. The computational models suggest that these
results align most naturally with mechanisms based on
procedural or associative learning. This indicates that
even in simple rule-based tasks, criterial learning may
be driven by associative processes.

Given the importance of criterial learning across various
decision models and the current gaps in understanding its
cognitive and neural mechanisms, we believe this work will
be of broad interest to the journal's readership.

For expert evaluation, we recommend the following reviewers
who possess expertise in criterial learning, category
learning, and computational modeling:

\begin{itemize}
    \item Dr. Carol A. Seger; \href{mailto:Carol.Seger@colostate.edu}{Carol.Seger@colostate.edu}
    \item Dr. S\'ebastien H\'elie; \href{mailto:shelie@purdue.edu}{shelie@purdue.edu}
    \item Dr. Corey Bohil \href{mailto:cbohil@ltu.edu}{cbohil@ltu.edu}
\end{itemize}

We affirm that this manuscript is original, has not been
previously published, and is not under concurrent
consideration elsewhere. All data and materials necessary
for reproducing the reported results are available in our
GitHub repository:
\url{https://github.com/crossley/crit_learn_delay}.

We appreciate your consideration of our manuscript for
publication in \textit{JEP: Learning, Memory, and Cognition}
and look forward to your feedback.

Sincerely,

Matthew J. Crossley\footnote{Corresponding author: matthew.crossley@mq.edu.au} \\
matthew.crossley@mq.edu.au \\
School of Psychological Sciences \\
Macquarie University Performance and Expertise Centre \\
Australian Hearing Hub \\
16 University Ave \\
Macquarie University, NSW 2109, Australia \\

Benjamin O. Pelzer \\
neurobot01@gmail.com \\
Independent Researcher \\
J\"ulich, North Rhine-Westphalia, Germany \\

F. Gregory Ashby \\
ashby@psych.ucsb.edu \\
Department of Psychological and Brain Sciences \\
251 University of California Santa Barbara \\
Santa Barbara, CA 93106, USA \\

\end{document}
